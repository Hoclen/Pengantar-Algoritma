\chapter{Ad Hoc 2}

\begin{permasalahan}{Pengurutan Kata}\\
Diberikan sejumlah kata dengan panjang yang sama. Kata-kata tersebut diurutkan berdasarkan urutan alphabet.\\
	\\
	\textbf{Masukan}\\
	Baris pertama merupakan jumlah kata yang akan diurutkan. Baris-baris selanjutnya merupakan kata-kata yang akan diurutkan.\\
	\textbf{Keluaran}\\
	Kata-kata yang diurutkan dimana berdasarkan urutan alphabet.\\
	\begin{center}
	\textbf{Test Case 1}\\
	\end{center}
	\textbf{Masukan}\\
	3\\
House\\
Ghost\\
Mouse\\

	\textbf{Keluaran}\\
	Ghost\\
House\\
Mouse\\
	\begin{center}
	\textbf{Test Case 2}\\
	\end{center}
	\textbf{Masukan}\\
	2\\
Flower\\
Flavor\\

	\textbf{Keluaran}\\
Flavor\\
Flower\\
	\begin{center}
	\textbf{Test Case 3}\\
	\end{center}
	\textbf{Masukan}\\
	4\\
Die\\
Day\\
Can\\
Gun\\

	\textbf{Keluaran}\\
Can\\
Day\\
Die\\
Gun\\
\end{permasalahan}

\newpage

\begin{permasalahan}{Pengurutan Tanggal}\\
Diberikan sejumlah tanggal dengan panjang yang sama. Tanggal-tanggal tersebut diurutkan berdasarkan urutan waktu.\\
	\\
	\textbf{Masukan}\\
	Baris pertama merupakan jumlah tanggal yang akan diurutkan. Baris-baris selanjutnya merupakan tanggal-tanggal yang akan diurutkan. Tanggal-tanggal tersebut ditulis dalam format DD MMM YYYY.\\
	\textbf{Keluaran}\\
	Tanggal-tanggal yang diurutkan berdasarkan urutan waktu.\\
	\begin{center}
	\textbf{Test Case 1}\\
	\end{center}
	\textbf{Masukan}\\
	3\\
06 Jun 2008\\
17 Agu 2010\\
12 Apr 2007\\

	\textbf{Keluaran}\\
12 Apr 2007\\
06 Jun 2008\\
17 Agu 2010\\
	\begin{center}
	\textbf{Test Case 2}\\
	\end{center}
	\textbf{Masukan}\\
5\\
03 Jan 1991\\
05 Jun 2004\\
04 Sep 2005\\
14 Feb 1933\\
27 Mar 1967\\

	\textbf{Keluaran}\\
14 Feb 1933\\
27 Mar 1967\\
03 Jan 1991\\
05 Jun 2004\\
04 Sep 2005\\
	\begin{center}
	\textbf{Test Case 3}\\
	\end{center}
	\textbf{Masukan}\\
2\\
17 Feb 2010\\
28 Feb 2003\\

	\textbf{Keluaran}\\
28 Feb 2003\\
17 Feb 2010\\
\end{permasalahan}

\newpage
\begin{permasalahan}{Pengurutan karakter berdasarkan kode ASCII}\\
Diberikan sejumlah karakter dan diurutkan secara ascending berdasarkan kode ASCII.\\
	\\
	\textbf{Masukan}\\
	String yang terdiri dari sejumlah karakter yang akan diurutkan.\\
	\textbf{Keluaran}\\
	String yang terdiri dari sejumlah karakter yang telah diurutkan.\\
	\begin{center}
	\textbf{Test Case 1}\\
	\end{center}
	\textbf{Masukan}\\
ertdf3\\

	\textbf{Keluaran}\\
3defrt\\
	\begin{center}
	\textbf{Test Case 2}\\
	\end{center}
	\textbf{Masukan}\\
*\&()as\\

	\textbf{Keluaran}\\
\&()*as\\
	\begin{center}
	\textbf{Test Case 3}\\
	\end{center}
	\textbf{Masukan}\\
kh5uOPr\^\\

	\textbf{Keluaran}\\
5OP\^{}hkru\\
\end{permasalahan}