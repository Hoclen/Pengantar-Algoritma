\chapter{Perulangan dan Array}

\section{Petunjuk}
\begin{itemize}
	\item Perhatikan petunjuk Dosen mengenai beda Latihan dan Permasalahan !
	\
	\item Perhatikan dan ikuti petunjuk dosen mengenai bagaimana cara menggunakan program Judge untuk mengevaluasi permasalahan yang Anda kerjakan!
\end{itemize}

\section{Latihan}


\begin{latihan}
Ketikkan dan jalankan listing \ref{lst:anakAyam1} dan \ref{lst:anakAyam2} berikut pada dua file berbeda ! 
\lstinputlisting[language=Python,
									label={lst:anakAyam1},
									caption=Lagu Anak Ayam (for)]									
 							  {code/AnakAyam.py}



\lstinputlisting[language=Python,
									label={lst:anakAyam2},
									caption=Lagu Anak ayam (while)]									
 							  {code/AnakAyam2.py}
								
Bandingkan kedua hasil listing diatas ! 
\end{latihan}

\begin{latihan}
Ketikkan dan jalankan listing \ref{lst:polaAngka1}  berikut  ! 
\lstinputlisting[language=Python,
									label={lst:polaAngka1},
									caption=Lagu Anak Ayam (for)]									
 							  {code/PolaAngka1.py}
Dapatkah Anda menerangkan apa tujuan algoritma diatas ?  
\end{latihan}


\begin{latihan}
Ketikkan dan jalankan listing \ref{lst:polaBintang1}  berikut  ! 
\lstinputlisting[language=Python,
									label={lst:polaBintang1},
									caption=Pola Bintang (for bersarang)]									
 							  {code/PolaBintang1.py}
Dapatkah Anda menerangkan apa tujuan algoritma diatas ?  



\end{latihan}



\newpage
\section{Permasalahan}
\begin{permasalahan}{Permasalahan Bilangan Kelipatan}\\
\label{prob:bilanganKelipatan}
	Hasilkan serangkaian $n$ bilangan yang merupakan kelipatan dari angka $m$ yang dimasukkan.\\
	\textbf{Masukan}\\
	Dua baris dimana baris pertama adalah $n$ dan baris kedua adalah $m$. $n$ merupakan panjang rangkaian yang akan dihasilkan sedangkan $m$ adalah kelipatan bilangan yang diinginkan. $n$ dan $m$ adalah bilangan bulat\\
	\textbf{Keluaran}\\
	Satu set rangkaian bilangan bulat dengan panjang rangkaian $n$ dan merupakan kelipatan dari $m$. Semua rangkaian bermula dari angka $m$.\\
	\begin{center}
	\textbf{Test Case 1}\\
	\end{center}
	\textbf{Masukan}\\
	6\\
	5\\
	\textbf{Keluaran}\\
	5 10 15 20 25 30 \\
	\begin{center}
	\textbf{Test Case 2}\\
	\end{center}
	\textbf{Masukan}\\
	3\\
	4\\
	\textbf{Keluaran}\\
	4 8 12 \\
	\begin{center}
	\textbf{Test Case 3}\\
	\end{center}
	\textbf{Masukan}\\
	10\\
	2\\
	\textbf{Keluaran}\\
	2 4 6 8 10 12 14 16 18 20\\
\end{permasalahan}

%\begin{permasalahan}
%Cetak Pola Bintang
%\end{permasalahan}
%
%\begin{permasalahan}
%Cetak Pola Deret
%\end{permasalahan}
%
%\begin{permasalahan}
%Diberikan String
%Ambil Huruf Awal, Tengah dan Akhir
%\end{permasalahan}
%
%\begin{permasalahan}
%Huruf - Huruf apa saja yang ada dari kalimat ? Musti Distinct
%\end{permasasalahan}
%
%\begin{permasalahan}
%Ada berapa angka yang dicari dari kumpulan angka yang berdempetan satu sama lain
%\end{permasasalahan}
%
%\begin{permasalahan}
%Ada berapa angka yang dicari dari kumpulan angka yang berdempetan satu sama lain
%\end{permasasalahan}
%
%\begin{permasalahan}
%
%\end{permasasalahan}
%
%\begin{permasalahan}
%double_char('The') → 'TThhee'
%double_char('AAbb') → 'AAAAbbbb'
%double_char('Hi-There') → 'HHii--TThheerree'
%\end{permasasalahan}


%\begin{panduan}{Tes}
%\begin{enumerate}
	%\item Baca Permasalahan \ref{prob:bilanganKelipatan}.
	%\item Buka Pyscripter.
	%\item Ketikkan Listing \ref{lst:permasalahan1}.
	%\begin{listprog}{Permasalahan 1 (permasalahan1.py)}
		%\label{lst:permasalahan1}
		%\begin{lstlisting}[language=Python]
		%n = input()
		%m = input()
		%for i in range(1,n+1):
	    %print m*i,
		%\end{lstlisting}
	%\end{listprog}
	%\item \textit{Save} file tersebut sebagai permasalahan1.py
	%\item Masuk ke http://elearning.mikroskil.ac.id, dan masuk ke \textit{Course} Pengantar Algoritma.
	%\item Klik di \textbf{Tugas Praktek 1: Permasalahan Bilangan Kelipatan}.
	%\item Klik Browse.
	%\item Pilih file permasalahan1.py yang sudah anda \textit{save} dan klik ok.
	%\item Refresh Browser sampai tulisan \textbf{Status} dari Pending menjadi Accepted. Jika ada error berarti ada kesalahan. Cek kembali Listing \ref{lst:permasalahan1}.
%\end{enumerate}
%\end{panduan}


%
%\newpage
%\begin{permasalahan}{Permasalahan Penjumlahan Bilangan Ganjil}\\
	%Diberikan sebuah bilangan $n$ carilah jumlah dari semua bilangan ganjil antara 0 sampai dengan bilangan $n$ tersebut.\\
	%\textbf{Masukan}\\
	%Sebuah bilangan bulat $n$.\\
	%\textbf{Keluaran}\\
	%Sebuah bilangan bulat $z$ dimana $z$ merupakan hasil penjumlahan dari semua bilangan ganjil yang ada antara 0 sampai bilangan $n$ tersebut (bilangan $n$ termasuk).\\
	%\begin{center}
	%\textbf{Test Case 1}\\
	%\end{center}
	%\textbf{Masukan}\\
	%20\\
	%\textbf{Keluaran}\\
	%100\\
	%\textit{Penjelasan: jumlah bilangan ganjil antara 1 sampai dengan 20 \\adalah 1+3+5+7+9+11+13+49=100}\\
	%\begin{center}
	%\textbf{Test Case 2}\\
	%\end{center}
	%\textbf{Masukan}\\
	%50\\
	%\textbf{Keluaran}\\
	%625\\
	%\begin{center}
	%\textbf{Test Case 3}\\
	%\end{center}
	%\textbf{Masukan}\\
	%90\\
	%\textbf{Keluaran}\\
	%2025\\
%\end{permasalahan}
%

