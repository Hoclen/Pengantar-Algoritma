\chapter{Percabangan}

\section{Petunjuk}
\begin{itemize}
	\item Perhatikan petunjuk Dosen untuk masalah Percabangan!
	\
	\item Perhatikan dan ikuti petunjuk dosen mengenai bagaimana cara menggunakan program Judge untuk mengevaluasi permasalahan yang Anda kerjakan!

\end{itemize}

\newpage
\section{Permasalahan}
\begin{itemize}

		\item Hubungi Dosen atau Asisten Dosen jika Anda terdapat \textit{Permasalahan} yang Anda sudah yakin benar, namun tetap mendapatkan \textit{Wrong Answer}. 
		\item \textbf{Periksa kembali berulang - ulang input, proses dan ouput Anda sebelum menghubungi Dosen atau Asisten Dosen}
		
		\item Anda akan dianggap menjawab satu sesi permasalahan jika telah mengkonfirmasi kode sumber Anda dengan seluruh \textit{test set}. Misal : Permasalahan 2-1 dianggap benar jika \textit{submission} Anda terhadap \textit{problem 2-1a, 2-1b, 2-1c ACCEPTED}
		
\end{itemize}


\begin{permasalahan}{Dimana Posisi Ku ?}\\
\label{prob:dimanaPosisiKu}
Tentukanlah nilai sebenarnya dari sebuah titik yang diberikan nilai $x$ dan $y
$ nya jika diberitahukan letak kuadran cartesian-nya ! \\
Misalkan : Titik ( 3,4) \\
Kuadran I : 3,4 \\
Kuadran II: -3,4 \\
Kuadran III : -3,-4 \\
Kuadran IV : 3,-4 \\\\
	\textbf{Masukan}\\
	2 buah input bilangan bulat positif \\
	1 buah input string yang berisi : \textbf{I, II, III, IV}\\\\
	\textbf{Keluaran}\\
	Koordinat titik dengan format (x,y) \\
	\begin{center}
	\textbf{Test Case}\\
	\end{center}
	\textbf{Masukan}\\
	3 \\
	4 \\
	III \\\\
	\textbf{Keluaran}\\
	(-3,-4)
\end{permasalahan}

\newpage
\begin{permasalahan}{30 Februari 2014}\\
\label{prob:30Februari}
Dapatkah Anda memberitahukan kepada saya apakah 30 - 2 - 2014 adalah tanggal yang benar ?  Sekaligus juga 31 - 4 - 2002 ? Dapatkah Anda memberitahu validitasnya setiap saya bertanya ? \\\\
	\textbf{Masukan}\\
	3 buah input bilangan bulat positif berurutan berupa tanggal, bulan dan tahun\\
	tanggal dapat berupa bilangan bulat positif dan negatif\\
	bulan dapat berupa bilangan bulat positif dan negatif\\
	tahun dapat berupa bilangan bulat positif dan negatif\\\\
	\textbf{Keluaran}\\
	Keluaran berupa keterangan dengan format :\\
	(tanggal) (nama bulan) (tahun) (validitas)\\
	\textbf{(tanggal)} : bilangan bulat positif / negatif \\
	\textbf{(nama bulan)} : string jika nama bulan ada (lower case). bilangan bulat positif / negatif jika nama bulan tidak ada\\
	\textbf{(tahun)} : bilangan bulat positif / negatif\\
	\textbf{(validitas)} : keterangan berupa ``valid`` atau ``tidak valid``\\

	\textbf{UKURAN VALIDITAS : }\\
	tanggal valid jika sesuai dengan jumlah tanggal setiap bulan, hati - hati tahun kabisat ! \\
	bulan valid jika masih diantara 1 - 12 \\
	tahun valid jika dari 1900 - 2099 \\

	\begin{center}
	\textbf{Test Case 1}\\
	\end{center}
	\textbf{Masukan}\\
	31 \\
	1 \\
	2012 \\\\
	\textbf{Keluaran}\\
	31 januari 2012 valid \\\\
	
	\begin{center}
	\textbf{Test Case 2}\\
	\end{center}
	\textbf{Masukan}\\
	29 \\
	2 \\
	2012 \\\\
	\textbf{Keluaran}\\
	29 februari 2012 valid \\\\

	\begin{center}
	\textbf{Test Case 3}\\
	\end{center}
	\textbf{Masukan}\\
	29 \\
	13 \\
	2012 \\\\
	\textbf{Keluaran}\\
	29 13 2012 tidak valid
	
	\begin{center}
	\textbf{Test Case 4}\\
	\end{center}
	\textbf{Masukan}\\
	29 \\
	2 \\
	2013 \\\\
	\textbf{Keluaran}\\
	29 februari 2013 tidak valid \\
	
	\begin{center}
	\textbf{Test Case 5}\\
	\end{center}
	\textbf{Masukan}\\
	29 \\
	1 \\
	2300 \\\\
	\textbf{Keluaran}\\
	29 januari 2300 tidak valid \\
	
\end{permasalahan}


\newpage
\begin{permasalahan}{Pythagoras}\\
\label{prob:pythagoras}
Apakah aku Segitiga Siku-siku atau Sembarang ? Dapatkah kamu menebaknya ? \\ 
Segitiga siku siku memiliki ciri - ciri : \begin{math}a^{2} + b^{2} = c^{2}\end{math}\\\\
	\textbf{Masukan}\\
	3 buah bilangan bulat \textbf{POSITIF (Tanpa negatif)} berupa tiga sisi segitiga\\
	\textbf{Tidak ada input yang salah !}\\\\
	\textbf{Keluaran}\\
	Keluaran berupa keterangan dengan format :\\
	(sisiA) (sisiB) (sisiC) (hasil)\\	
	\textbf{(sisi a)} : sisi pembentuk sudut siku - siku
	\textbf{(sisi b)} : sisi pembentuk sudut siku - siku
	\textbf{(sisi c)} : sisi miring bilangan bulat positif 
	\textbf{(hasil)} : keterangan berupa ``segitiga siku-siku`` atau ``segitiga siku-siku``\\

	sisi a, sisi b, dan sisi c adalah bilangan bulat positif. 
	sisi a dan sisi b harus berurutan

	\begin{center}
	\textbf{Test Case 1}\\
	\end{center}
	\textbf{Masukan}\\
	3 \\
	4 \\
	5 \\\\
	\textbf{Keluaran}\\
	3 4 5 segitiga siku-siku \\\\
	
	\begin{center}
	\textbf{Test Case 2}\\
	\end{center}
	\textbf{Masukan}\\
	4 \\
	5 \\
	3 \\\\
	\textbf{Keluaran}\\
	3 4 5 segitiga siku-siku \\\\

	\begin{center}
	\textbf{Test Case 3}\\
	\end{center}
	\textbf{Masukan}\\
	5 \\
	4 \\
	3 \\\\
	\textbf{Keluaran}\\
	3 4 5 segitiga siku-siku \\\\
	
	\begin{center}
	\textbf{Test Case 4}\\
	\end{center}
	\textbf{Masukan}\\
	141 \\
	3 \\
	141 \\\\
	\textbf{Keluaran}\\
	3 141 141 bukan segitiga siku-siku \\	
\end{permasalahan}



\newpage
\begin{permasalahan}{OWNING !!!}\\
\label{prob:OWNING}
Defense of The Ancients(DOTA) merupakan salah satu permainan e-sport paling popular, di mana dua tim yang masing - masing terdiri dari 5 pemain bersaing untuk meruntuhkan pertahanan tim lawan dengan menghancurkan bangunan inti lawan, yang dikenal dengan nama “Ancient”. Dalam permainan ini, setiap kali pesaing hero mengeleminasi lawan, maka pemain akan mendapatkan gelar dan pujian.\\

Berikut adalah urutan gelar bedasarkan \textbf{total skor / eleminasi} :\\
         3 -  "killing spree" \\
         4 -  "dominating" \\ 
         5 - "mega kill" \\
         6 -  "unstoppable" \\
         7 - "wicked sick"
         8 -  "monster kill" \\
         9 -  "godlike" \\
         lebih dari 10 -  "beyond godlike" \\

Berikut adalah urutan pujian bedasarkan \textbf{jumlah eleminasi yang barusan terjadi} :\\
         2 -  "double kill" \\ 
         3 - "triple kill" \\
         4 -  "ultra kill" \\
				 lebih dari 5 - "rampage" \\
				
	Dapatkah Anda menentukan gelar dan pujian akan diucapkan Announcer jika diketahui jumlah eleminasi sebelumnya dan jumlah eleminasi baru ? \\\\
 
	\textbf{Masukan}\\
	2 buah bilangan bulat \textbf{POSITIF (Tanpa negatif)} berupa (jumlah eleminasi sebelumnya) dan (jumlah eleminasi baru) \\
	\textbf{Tidak ada input yang salah !}\\\\
	\textbf{Keluaran}\\
	Keluaran berupa keterangan dengan format :\\
	(total eleminasi:bilangan bulat positif) (gelar:string) (pujian:string) \\
	
	
	ketiadaan (gelar) dan (pujian) akan diwakili dengan ``n/a``

	\begin{center}
	\textbf{Test Case 1}\\
	\end{center}
	\textbf{Masukan}\\
	0 \\
	0 \\
	\textbf{Keluaran}\\
	0 0 n/a n/a\\\\
	
	\begin{center}
	\textbf{Test Case 2}\\
	\end{center}
	\textbf{Masukan}\\
	5 \\
	5 \\\\
	\textbf{Keluaran}\\
	5 5 beyond godlike rampage \\\\


	\begin{center}
	\textbf{Test Case 3}\\
	\end{center}
	\textbf{Masukan}\\
	1 \\
	4 \\\\
	\textbf{Keluaran}\\
	1 4 mega kill ultra kill \\\\
	
	\begin{center}
	\textbf{Test Case 4}\\
	\end{center}
	\textbf{Masukan}\\
	3 \\
	2 \\\\
	\textbf{Keluaran}\\
	3 2 mega kill double kill \\	
\end{permasalahan}



\newpage
\begin{permasalahan}{Twenty-Nine My Age}\\
\label{prob:a29myage}
Vicky mengerti bahasa Indonesia tetapi hanya bisa dapat menjawab dalam bahasa Inggris. Dapatkah kamu membuat algoritma yang mirip dengan dirinya, dimana jika diberikan hari dalam bahasa indonesia dan jumlah hari yang akan ditambahkan dia akan menjawab dengan hari dalam bahasa inggris serta jarak hari\\\\
 
	\textbf{Masukan}\\
	1 buah string berupa hari dalam minggu (bahasa Indonesia)\\
	1 buah bilangan bulat \textbf{POSITIF (Tanpa negatif)} berupa jumlah hari kedepan \\
	\textbf{Tidak ada input yang salah !}\\\\
	\textbf{Keluaran}\\
	Keluaran berupa keterangan dengan format :\\
	(hari awal) (banyak hari) (hari akhir) (jarak hari)\\
	- (hari awal) dalam bahasa Indonesia\\
	- (banyak hari) yang akan ditambahkan : bilangan bulat positif\\
	- (hari akhir) dalam bahasa Inggris \\
	- (jarak hari) antara hari antara hari awal dengan hari akhir : bilangan bulat positif / negatif \\

	\begin{center}
	\textbf{Test Case 1}\\
	\end{center}
	\textbf{Masukan}\\
	senin \\
	14 \\
	\textbf{Keluaran}\\
	senin 14 monday 0\\\\
	
	\begin{center}
	\textbf{Test Case 2}\\
	\end{center}
	\textbf{Masukan}\\
	kamis \\
	141 \\\\
	\textbf{Keluaran}\\
	kamis 141 friday -1 \\\\


	\begin{center}
	\textbf{Test Case 3}\\
	\end{center}
	\textbf{Masukan}\\
	rabu \\
	18 \\\\
	\textbf{Keluaran}\\
	rabu 18 sunday 3 \\\\
	
	\begin{center}
	\textbf{Test Case 4}\\
	\end{center}
	\textbf{Masukan}\\
	jumat \\
	70 \\\\
	\textbf{Keluaran}\\
	jumat 70 friday 0 \\	
\end{permasalahan}
