\chapter{Percabangan}

\begin{permasalahan}{Permasalahan Bilangan Kelipatan}\\
\label{prob:bilanganKelipatan}
	Hasilkan serangkaian $n$ bilangan yang merupakan kelipatan dari angka $m$ yang dimasukkan.\\
	\textbf{Masukan}\\
	Dua baris dimana baris pertama adalah $n$ dan baris kedua adalah $m$. $n$ merupakan panjang rangkaian yang akan dihasilkan sedangkan $m$ adalah kelipatan bilangan yang diinginkan. $n$ dan $m$ adalah bilangan bulat\\
	\textbf{Keluaran}\\
	Satu set rangkaian bilangan bulat dengan panjang rangkaian $n$ dan merupakan kelipatan dari $m$. Semua rangkaian bermula dari angka $m$.\\
	\begin{center}
	\textbf{Test Case 1}\\
	\end{center}
	\textbf{Masukan}\\
	6\\
	5\\
	\textbf{Keluaran}\\
	5 10 15 20 25 30 \\
	\begin{center}
	\textbf{Test Case 2}\\
	\end{center}
	\textbf{Masukan}\\
	3\\
	4\\
	\textbf{Keluaran}\\
	4 8 12 \\
	\begin{center}
	\textbf{Test Case 3}\\
	\end{center}
	\textbf{Masukan}\\
	10\\
	2\\
	\textbf{Keluaran}\\
	2 4 6 8 10 12 14 16 18 20\\

\end{permasalahan}

\begin{panduan}{Tes}
\begin{enumerate}
	\item Baca Permasalahan \ref{prob:bilanganKelipatan}.
	\item Buka Pyscripter.
	\item Ketikkan Listing \ref{lst:permasalahan1}.
	\begin{listprog}{Permasalahan 1 (permasalahan1.py)}
		\label{lst:permasalahan1}
		\begin{lstlisting}[language=Python]
		n = input()
		m = input()
		for i in range(1,n+1):
	    print m*i,
		\end{lstlisting}
	\end{listprog}
	\item \textit{Save} file tersebut sebagai permasalahan1.py
	\item Masuk ke http://elearning.mikroskil.ac.id, dan masuk ke \textit{Course} Pengantar Algoritma.
	\item Klik di \textbf{Tugas Praktek 1: Permasalahan Bilangan Kelipatan}.
	\item Klik Browse.
	\item Pilih file permasalahan1.py yang sudah anda \textit{save} dan klik ok.
	\item Refresh Browser sampai tulisan \textbf{Status} dari Pending menjadi Accepted. Jika ada error berarti ada kesalahan. Cek kembali Listing \ref{lst:permasalahan1}.
\end{enumerate}
\end{panduan}

\newpage
\begin{permasalahan}{Permasalahan Bilangan Genap Atau Ganjil}\\
	Cari tahu apakah sebuah bilangan merupakan bilangan genap atau ganjil.\\
	\textbf{Masukan}\\
	Sebuah bilangan $n$ yang merupakan bilangan bulat.\\
	\textbf{Keluaran}\\
	Keluaran antara dua bilangan saja, yaitu 1 apabila bilangan genap, atau, 2 apabilan bilangan ganjil.\\
	\begin{center}
	\textbf{Test Case 1}\\
	\end{center}
	\textbf{Masukan}\\
	5\\
	\textbf{Keluaran}\\
	2\\
	\begin{center}
	\textbf{Test Case 2}\\
	\end{center}
	\textbf{Masukan}\\
	98472\\
	\textbf{Keluaran}\\
	1\\
	\begin{center}
	\textbf{Test Case 3}\\
	\end{center}
	\textbf{Masukan}\\
	6536\\
	\textbf{Keluaran}\\
	1\\
\end{permasalahan}

\newpage
\begin{permasalahan}{Permasalahan Penjumlahan Bilangan Ganjil}\\
	Diberikan sebuah bilangan $n$ carilah jumlah dari semua bilangan ganjil antara 0 sampai dengan bilangan $n$ tersebut.\\
	\textbf{Masukan}\\
	Sebuah bilangan bulat $n$.\\
	\textbf{Keluaran}\\
	Sebuah bilangan bulat $z$ dimana $z$ merupakan hasil penjumlahan dari semua bilangan ganjil yang ada antara 0 sampai bilangan $n$ tersebut (bilangan $n$ termasuk).\\
	\begin{center}
	\textbf{Test Case 1}\\
	\end{center}
	\textbf{Masukan}\\
	20\\
	\textbf{Keluaran}\\
	100\\
	\textit{Penjelasan: jumlah bilangan ganjil antara 1 sampai dengan 20 \\adalah 1+3+5+7+9+11+13+49=100}\\
	\begin{center}
	\textbf{Test Case 2}\\
	\end{center}
	\textbf{Masukan}\\
	50\\
	\textbf{Keluaran}\\
	625\\
	\begin{center}
	\textbf{Test Case 3}\\
	\end{center}
	\textbf{Masukan}\\
	90\\
	\textbf{Keluaran}\\
	2025\\
\end{permasalahan}

\newpage
\begin{permasalahan}{Permasalahan Pengecekkan Bilangan Prima}\\
	Diberikan sebuah bilangan $n$ cari tahu apakah itu bilangan prima atau bukan.\\
	\textbf{Masukan}\\
	Sebuah bilangan $n$ dimana $n$ adalah bilangan integer.\\
	\textbf{Keluaran}\\
	Keluaran bisa berupa dua yaitu 1 dan 0. Angka 1 menandakan bahwa itu adalah Prima, sedangkan angka 0 menandakan itu bukan bukan bilangan Prima.\\
	\begin{center}
	\textbf{Test Case 1}\\
	\end{center}
	\textbf{Masukan}\\
	3\\
	\textbf{Keluaran}\\
	1\\
	
	\begin{center}
	\textbf{Test Case 2}\\
	\end{center}
	\textbf{Masukan}\\
	26\\
	\textbf{Keluaran}\\
	0\\
	
	\begin{center}
	\textbf{Test Case 3}\\
	\end{center}
	\textbf{Masukan}\\
	11\\
	\textbf{Keluaran}\\
	1\\
	
\end{permasalahan}

\newpage
\begin{permasalahan}{Permasalahan Pencarian Bilangan Prima}\\
	Hasilkan serangkaian bilangan prima dari bilangan $2$ sampai dengan bilangan $n$.\\
	\textbf{Masukan}\\
	Sebuah bilangan bulat \textit{n} yang merupakan batas atas dari \textit{array} bilangan prima yang akan dihasilkan.\\
	\textbf{Keluaran}\\
	Satu set (himpunan elemen yang tidak memiliki duplikat) bilangan $A$ yang terdiri dari bilangan $2$ sampai bilangan $n$ (termasuk $n$) dimana setiap bilangan hanya memiliki dua pembagi saja yaitu $1$ dan bilangan itu sendiri.\\
	\begin{center}
	\textbf{Test Case 1}
	\end{center}
	\textbf{Masukan}
	10
	\textbf{Keluaran}
	2 3 5 7
	\begin{center}
	\textbf{Test Case 2}
	\end{center}
	\textbf{Masukan}
	19
	\textbf{Keluaran}
	2 3 5 7 11 13 17 19
	\begin{center}
	\textbf{Test Case 3}
	\end{center}
	\textbf{Masukan}
	25
	\textbf{Keluaran}
	2 3 5 7 11 13 17 19 23 
\end{permasalahan}

