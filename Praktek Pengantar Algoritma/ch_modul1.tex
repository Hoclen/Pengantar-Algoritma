\chapter{Pengenalan Algoritma \& Pemrograman dengan Python}

\section{Petunjuk}
\begin{itemize}
	\item Perhatikan petunjuk Dosen mengenai beda Latihan dan Permasalahan !
	\
	\item Perhatikan dan ikuti petunjuk dosen mengenai bagaimana cara menggunakan program Judge untuk mengevaluasi permasalahan yang Anda kerjakan!
\end{itemize}

\section{Latihan}
\begin{latihan}
Hitunglah dengan menggunakan python langkah-langkah berikut:
\begin{enumerate}
	\item Isi $a$ dengan 300
	\item Isi $b$ dengan 700
	\item Kalikan $a$ dengan $b$ dan taruh di dalam $c$
	\item Isi $d$ dengan 1200
	\item Tambahkan isi $a$ dengan 60 dan kemudian kalikan hasil tersebut dengan $10$ dan simpan kembali ke $a$   
	\item Bagikan $a$ dengan $d$ dan isi ke dalam $e$
	\item Tambahkan $b$ dengan $c$ dan isi ke dalam $c$
	\item Tambahkan $a$ dengan $e$ kemudian kurangkan dengan $c$ dan simpan ke $b$
\end{enumerate}
Berapakah hasil akhir variabel $a$, $b$, $c$, $d$, dan $e$? 
Tuliskan langkah-langkah yang anda ketikkan di secarik kertas.
\end{latihan}

\begin{latihan}
Ada lima variabel yaitu $a$ = 1, $b$ = 2, $c$ = 3, $d$ = 4, dan $e$ = 5. Tukarkan variabel tersebut sehingga hasilnya $a$ = 5, $b$ = 3, $c$ = 4, $d$ = 1, $e$ = 2. Tuliskan langkah-langkah yang anda ketikkan di secarik kertas. 
\end{latihan}


\begin{latihan}
Buatkan program yang meminta input untuk NIM, nama, jenis kelamin, dan umur anda. Kemudian tampilkan informasi berikut dengan sintaks print.
\end{latihan}

\begin{latihan}
Buatkan program yang bisa menghitung luas segitiga, luas segiempat dan luas lingkaran. Tentukan variabel apa yang perlu diinput dan tampilkan informasi luas dengan menggunakan sintaks print.
\end{latihan}

\begin{latihan}
\label{latihan:Gelas}
Lakukan langkah - langkah berikut ini pada IDLE : 
\begin{enumerate}
	\item GelasA = 130,85181
	\item GelasB = 194,12141
	\item print(GelasA)
	\item print(GelasB)
	\item print(GelasA,GelasB)
	\item print(``\%.2f`` \% GelasA,GelasB)
	\item print(``\%.2f`` \% GelasA,``\%.3f`` \% GelasB)
\end{enumerate}
\end{latihan}

\newpage
\section{Permasalahan}
\begin{permasalahan}{Cetak Mania}\\
\label{prob:CetakMania}
Anda adalah mahasiswa baru Teknik Informatika. Anda berniat membuat blog yang mewakili Nama Anda dan tahun lahir Anda.\\
Catatan : Tanda `\textbackslash` adalah karakter spesial. Gunakan `\textbackslash\textbackslash` untuk mencetak sebuah karakter `\textbackslash` \\
	\textbf{Masukan}\\
	4 buah String Tanpa Spasi yang mewakili Nama, Tanggal, Bulan dan Tahun Lahir \\
	\textbf{Keluaran}\\
	Alamat Blog Yang terdapat Nama dan Keterangan Lahir Anda \\
	\begin{center}
	\textbf{Test Case}\\
	\end{center}
	\textbf{Masukan}\\
	Budi \\
	15 \\
	12 \\
	1990 \\
	\textbf{Keluaran}\\
	http:\textbackslash\textbackslash www.blogbuatanku.co.id\textbackslash Budi1512.1990 \\	
\end{permasalahan}

\newpage
\begin{permasalahan}{Menghitung Luas dan Volume Bola}\\
\label{prob:Volume}
Hitunglah luas dan volume bola jika diketahui nilai jari - jari R dari bola. \\
Keterangan : \\
  \ \\
		\begin{math}
				Luas = 4\ \times\ \frac{22}{7}\ \times\ $jari-jari$^{2} \\
				Volume = \frac{4}{3}\ \times\ \frac{22}{7}\  \times\ $jari-jari$^{3} \\
		\end{math}
   \ \\
	\textbf{Masukan}\\
	Sebuah bilangan $R$ yang merupakan bilangan pecahan\\
	\textbf{Keluaran}\\
Keluaran berupa dua bilangan pecahan yang secara berurut adalah Luas dan Volume Bola. Bilangan pecahan output harus memiliki desimal maksimum tiga dibelakang koma. Triknya adalah menggunakan perintah  \textit{print}(``\%.2f`` \% \textit{namaVariabel}) sama seperti Latihan \ref{latihan:Gelas}
	\begin{center}
	\textbf{Test Case}\\
	\end{center}
	\textbf{Masukan}\\
	7\\
	\textbf{Keluaran}\\
	616.000 1437.333\\	
\end{permasalahan}

\newpage
\begin{permasalahan}{Petualangan Semut}\\
\label{prob:PetualanganSemut}
	Seekor semut menempuh perjalanan sejauh x cm. Buat algoritma \& program untuk mengkonversi jarak x cm dalam km-m-cm. \\
	\textbf{Masukan}\\
	Sebuah bilangan $X$ yang merupakan bilangan bulat\\
	\textbf{Keluaran}\\
	Keluaran berupa tiga bilangan pecahan yang secara berurut dengan format $a$ km, $b$ m, $c$ cm
	\begin{center}
	\textbf{Test Case}\\
	\end{center}
	\textbf{Masukan}\\
	506141\\
	\textbf{Keluaran}\\
	5 km, 61 m, 41 cm
\end{permasalahan}

\newpage
\begin{permasalahan}{Konversi Suhu}\\
\label{prob:KonversiSuhu}
Budi tinggal di Amerika yang menggunakan satuan  Farenheit untuk mengukur suhu udara. Bantulah Budi untuk mengkonversi suhu menjadi satuan Celcius, Kelvin dan Reaumur. \\ 
	\textbf{Masukan}\\
	Sebuah bilangan $X$ yang merupakan bilangan bulat atau pecahan\\
	\textbf{Keluaran}\\
Keluaran berupa tiga bilangan pecahan yang secara berurut dengan $a$ Celcius, $b$ Kelvin, dan $c$ Reaumur satu dibelakang koma
 	\begin{center}
	\textbf{Test Case}\\
	\end{center}
	\textbf{Masukan}\\
	212\\
	\textbf{Keluaran}\\
	100.0 C\\
	373.0 K\\
	80.0 R
\end{permasalahan}

\newpage
\begin{permasalahan}{Persamaan}\\
\label{prob:Persamaan}
Diketahui 3 buah bilangan bulat $a$, $b$ dan $c$. Masukkan ke persamaan berikut: \\
	
		 $x = \frac{\neg b\ +2c^{2}\ +\ 4ab\ }{2c}$ \\
	
Hitung nilai $X$ \\
	\textbf{Masukan}\\
	Tiga buah bilangan bulat positif atau negatif \\
	\textbf{Keluaran}\\
	Bilangan pecahan dengan desimal dua dibelakang koma \\
 	\begin{center}
	\textbf{Test Case}\\
	\end{center}
	\textbf{Masukan}\\
	1\\
	2\\
	3\\
	\textbf{Keluaran}\\
	4.00\\
\end{permasalahan}

