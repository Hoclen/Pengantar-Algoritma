\chapter{Rekursif}

\begin{permasalahan}{Faktorial}\\
	Faktorial didefinisikan sebagai 1x2x3x4x$n$ untuk sebuah bilangan $n$. Diberikan sebuah bilangan $n$, tampilkan $n!$\\
	\textbf{Masukan}\\
	Sebuah bilangan bulat $n$.\\
	\textbf{Keluaran}\\
	Faktorial $n!$\\
	\begin{center}
	\textbf{Test Case 1}\\
	\end{center}
	\textbf{Masukan}\\
	4\\
	\textbf{Keluaran}\\
	24\\
	\begin{center}
	\textbf{Test Case 2}\\
	\end{center}
	\textbf{Masukan}\\
	10\\
	\textbf{Keluaran}\\
	3628800\\
	\begin{center}
	\textbf{Test Case 3}\\
	\end{center}
	\textbf{Masukan}\\
	12\\
	\textbf{Keluaran}\\
	479001600\\
\end{permasalahan}

\newpage
\begin{permasalahan}{Penjumlahan Bilangan Tribonacci}\\
	Bilangan Tribonacci merupakan deretan bilangan dimana bilangan tersebut merupakan hasil dari penjumlahan tiga bilangan pada urutan deret sebelumnya. Angka pertama,kedua pada deret bilangan Fibonacci adalah angka 1. Contoh deret adalah 1, 1, 1, 3, 5, 9, ... Tampilkan penjumlahan semua bilangan dalam rangkaian bilangan tribonacci $n$.
	\textbf{Masukan}\\
	Sebuah bilangan  bulat $n$ yang merupakan jumlah bilangan dalam rangkaian bilangan tribonacci.\\
	\textbf{Keluaran}\\
	Jumlah dari semua bilangan yang terdapat di dalam rangkaian bilangan tribonacci.\\
	\begin{center}
	\textbf{Test Case 1}\\
	\end{center}
	\textbf{Masukan}\\
	5\\
	\textbf{Keluaran}\\
	11\\
	\textit{Penjelasan: 1+1+1+3+5 = 11}\\
	\begin{center}
	\textbf{Test Case 2}\\
	\end{center}
	\textbf{Masukan}\\
	10\\
	\textbf{Keluaran}\\
	230\\
	\begin{center}
	\textbf{Test Case 3}\\
	\end{center}
	\textbf{Masukan}\\
	25\\
	\textbf{Keluaran}\\
	2145013\\
\end{permasalahan}