\chapter{Quiz Pertemuan VI}


\section{Permasalahan}
\begin{permasalahan}{Perkalian Genap}\\
\label{prob:PerkalianGenap}
	Lakukan pencarian hasil untuk perkalian bilangan genap sebanyak $N$ \\\\
	\textbf{Masukan}\\
	Sebuah bilangan bulat yang menyatakan terdapat $N$ bilangan\\
	\textbf{Keluaran}\\
	Hasil perkalian seluruh bilangan bulat genap sebanyak  $N$ bilangan
	\begin{center}
	\textbf{Test Case 1}\\
	\end{center}
	\textbf{Masukan}\\
	5\\
	\textbf{Keluaran}\\
	3840\\
	\begin{center}
	\textbf{Test Case 2}\\
	\end{center}
	\textbf{Masukan}\\
	8\\
	\textbf{Keluaran}\\
	10321920\\
\end{permasalahan}


\newpage
\begin{permasalahan}{Pola dan Penjumlahan}\\
\label{prob:PoladanPenjumlahan}
		Lakukan pencetakan pola dan pencarian hasil penjumlahan bilangan sebanyak $N$. Perhatikan bahwa pola memiliki memiliki perubahan dari positif dan negatif \\\\
	\textbf{Masukan}\\
	Sebuah bilangan bulat yang menyatakan terdapat $N$ bilangan\\
	\textbf{Keluaran}\\
	Pola \& Hasil penjumlahan seluruh bilangan sebanyak  $N$ bilangan
	\\
	\begin{center}
	\textbf{Test Case 1}\\
	\end{center}
	\textbf{Masukan}\\
	6\\
	\textbf{Keluaran}\\
1\\
-3\\
5\\
-7\\
9\\
-11\\
-6\\
	\begin{center}
	\textbf{Test Case 2}\\
	\end{center}
	\textbf{Masukan}\\
	11\\
	\textbf{Keluaran}\\
1\\
-3\\
5\\
-7\\
9\\
-11\\
13\\
-15\\
17\\
-19\\
21\\
11\\
\end{permasalahan}


\newpage
\begin{permasalahan}{Ganjil-Genap Genap-Ganjil ?}\\
\label{prob:GanjilGenap}
		Lakukan pembuatan terhadap pola sesuai dengan keluaran pada Test Case sebanyak N bilangan \\\\
	\textbf{Masukan}\\
	Sebuah bilangan bulat yang menyatakan terdapat $N$ bilangan\\
	\textbf{Keluaran}\\
	Pola sesuai dengan test case.
	\\
	\begin{center}
	\textbf{Test Case 1}\\
	\end{center}
	\textbf{Masukan}\\
	10\\
	\textbf{Keluaran}\\
2\\
1\\
4\\
3\\
6\\
5\\
8\\
7\\
10\\
9\\
	\begin{center}
	\textbf{Test Case 2}\\
	\end{center}
	\textbf{Masukan}\\
	11\\
	\textbf{Keluaran}\\
2\\
1\\
4\\
3\\
6\\
5\\
8\\
7\\
10\\
9\\
12\\
\end{permasalahan}






