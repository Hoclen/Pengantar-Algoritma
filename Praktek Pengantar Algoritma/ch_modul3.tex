\chapter{Pengurutan}

\begin{permasalahan}{Permasalahan Pengurutan Alphabet}\\
\label{prob:urutAlphabet}
	Diberikan sebuah kalimat yang terdiri dari huruf, urutkan semua alphabet secara menaik.\\
	\textbf{Masukan}\\
	Satu baris kalimat dimana terdapat huruf A-Z, dan a-z.\\
	\textbf{Keluaran}\\
	Urutkan kalimat tersebut sehingga setiap huruf tersortir sesuai dengan posisinya di alphabet. Huruf kapital dulan baru huruf non kapital\\
	\begin{center}
	\textbf{Test Case 1}\\
	\end{center}
	\textbf{Masukan}\\
	Selamat Datang Di Rumahku\\
	\textbf{Keluaran}\\
	D D R S a a a a a e g h i k l m m n t t u u\\
	\begin{center}
	\textbf{Test Case 2}\\
	\end{center}
	\textbf{Masukan}\\
	Hari ini hari yang cerah\\
	\textbf{Keluaran}\\
	H a a a a c e g h h i i i i n n r r r y\\
	\begin{center}
	\textbf{Test Case 3}\\
	\end{center}
	\textbf{Masukan}\\
	Halo halo halo\\
	\textbf{Keluaran}\\
	H a a a h h l l l o o o\\

\end{permasalahan}

\newpage
\begin{permasalahan}{Permasalahan Pengurutan Ganjil dan Genap}\\
	Diberikan sebuah bilangan, urutkan bilangan tersebut dan letakkan urutan bilangan genap duluan baru bilangan ganjil.\\
	\textbf{Masukan}\\
	Rangkaian bilangan bulat.\\
	\textbf{Keluaran}\\
	Urutan dari masukan bilangan tetapi di hasil urutan letakkan urutan bilangan genap duluan dan urutan bilangan ganjil belakangan (jika ada 0 di masukan bilangan maka selalu terletak di awal).\\
	\begin{center}
	\textbf{Test Case 1}\\
	\end{center}
	\textbf{Masukan}\\
	798123813734\\
	\textbf{Keluaran}\\
	2 4 8 8 1 1 3 3 3 7 7 9\\
	\textit{Penjelasan: jumlah bilangan ganjil antara 1 sampai dengan 20 \\adalah 1+3+5+7+9+11+13+49=100}\\
	\begin{center}
	\textbf{Test Case 2}\\
	\end{center}
	\textbf{Masukan}\\
	01841941747\\
	\textbf{Keluaran}\\
	0 4 4 4 8 1 1 1 7 7 9\\
	\begin{center}
	\textbf{Test Case 3}\\
	\end{center}
	\textbf{Masukan}\\
	532526211242\\
	\textbf{Keluaran}\\
	2 2 2 2 2 4 6 1 1 3 5 5\\
\end{permasalahan}



