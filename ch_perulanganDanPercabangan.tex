\chapter{Kombinasi Perulangan dan Percabangan (\textit{Looping})}


\section{Kombinasi Percabangan dan Perulangan}
Dalam kondisi tertentu, suatu perulangan dapat memiliki percabangan di dalamnya. Biasanya percabangan di dalam perulangan digunakan untuk mengkhususkan perintah berbeda yang akan dikerjakan ketika variabel mencapai nilai tertentu. Pada perulangan yang memiliki percabangan di dalamnya, mungkin saja terdapat perintah :
\begin{enumerate}
	\item \textit{break}, yang akan menghentikan perulangan walaupun nilai varibelnya belum melampaui batas.
	\item \textit{continue}, yang akan melanjutkan perulangan ke tahapan perulangan berikutnya dan tidak akan menjalankan perintah di bawahnya.
\end{enumerate}
Dua contoh berikut akan menunjukkannya.

\begin{listprog}{contohBreak.py}
	\label{lst:contohBreak}
	\begin{lstlisting}[language=Python]
	for i in range(1,6):
    if i==4:
        break
    print(i)
	#Maka yang dicetak adalah 1,2,3
	#Pada saat i mencapai nilai 4, perulangan langsung berhenti sebelum sempat mencetak

	\end{lstlisting}
\end{listprog}

\FloatBarrier
\begin{listprog}{contohContinue.py}
	\label{lst:contohContinue}
	\begin{lstlisting}[language=Python]
	for i in range(1,6):
    if i==4:
        continue
    print(i)
	#Maka yang dicetak adalah 1,2,3,5
	#Pada saat i mencapai nilai 4, perulangan langsung berhenti sebelum sempat mencetak

	\end{lstlisting}
\end{listprog}



\FloatBarrier
\begin{konsep}
\label{lat:pencetakanMatriks}
\textbf{Permasalahan pencetakan matriks}
Hasilkan sebuah algoritma dan \textit{flowchart} untuk mencetak matriks dengan ketentuan:
\begin{enumerate}
	\item Baris ganjil dari 1 sampai n
	\item Baris genap dari n turun sampai 1
\end{enumerate}
\textbf{Masukan}\\
Sebuah bilangan bulat $n$.\\ 
\textbf{Keluaran}\\
Keluarannya berupa matriks $n$x$n$ dengan baris ganjil merupakan rangkaian angka menaik dari 1 sampai n dan baris genap merupakan rangkaian angka menurun dari n sampai 1.\\
\begin{center}
\textbf{Test Case 1}\\
\end{center}
\textbf{Masukan}\\
5\\
\textbf{Keluaran}\\
1 2 3 4 5 \\
5 4 3 2 1 \\
1 2 3 4 5 \\
5 4 3 2 1 \\
1 2 3 4 5 \\
\begin{center}
\textbf{Test Case 2}\\
\end{center}
\textbf{Masukan}\\
3\\
\textbf{Keluaran}\\
1 2 3 \\
3 2 1 \\
1 2 3 \\
\end{konsep}

\begin{pemrograman}
Buatkan program python dari Latihan \ref{lat:pencetakanMatriks}.
\end{pemrograman}

\begin{konsep}
\label{lat:pencetakanBintang}
\textbf{Permasalahan pencetakkan bintang dari NIM}\\
Hasilkan algoritma dan \textit{flowchart} untuk mencetak bintang dan angka berikut dengan menggunakan NIM sebagai dasarnya.\\
\textbf{Masukan}\\
Sebuah \textit{String} 9 karakter dimana merupakan NIM dari mahasiswa STMIK Mikroskil.\\
\textbf{Keluaran}\\
9 baris dimana di baris i terdapat karakter i, satu spasi dan diikuti bintang sejumlah karakter i tersebut.\\
\begin{center}
\textbf{Test Case 1}\\
\end{center}
\textbf{Masukan}\\
121114567\\
\textbf{Keluaran}\\
1 * \\
2 ** \\
1 * \\
1 * \\
1 * \\
0 \\
5 ***** \\
6 ****** \\
7 ******* \\
\begin{center}
\textbf{Test Case 2}\\
\end{center}
\textbf{Masukan}\\
031110023\\
\textbf{Keluaran}\\
0  \\
3 *** \\
1 * \\
1 * \\
1 * \\
0 \\
0 \\
2 ** \\
3 *** \\
\end{konsep}

\begin{pemrograman}
Buatkan program python dari Latihan \ref{lat:pencetakanBintang}.
\end{pemrograman}